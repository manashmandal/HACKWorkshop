% 	Name		:: 	sthlm Beamer Theme  HEAVILY based on the hsrmbeamer theme (Benjamin Weiss)
%	Author		:: 	Mark Hendry Olson (mark@hendryolson.com)
%	Created		::	2013-07-31
%	Updated		::	June 18, 2015 at 08:45
%	Version		:: 	1.0.2
%	Email		:: 	hendryolson@gmail.com
%	Website		:: 	http://v42.com
%
% 	License		:: 	This file may be distributed and/or modified under the
%                  	GNU Public License.
%
%	Description	::	This presentation is a demonstration of the sthlm beamer
%					theme, which is HEAVILY based on the HSRM beamer theme created by Benjamin Weiss
%					(benjamin.weiss@student.hs-rm.de), which can be found on GitHub
%					<https://github.com/hsrmbeamertheme/hsrmbeamertheme>.


%-=-=-=-=-=-=-=-=-=-=-=-=-=-=-=-=-=-=-=-=-=-=-=-=
%
%        LOADING DOCUMENT
%
%-=-=-=-=-=-=-=-=-=-=-=-=-=-=-=-=-=-=-=-=-=-=-=-=

\documentclass[newPxFont]{beamer}
\usetheme{sthlm}
\usepackage{listings}
%\usecolortheme{sthlmv42}

%-=-=-=-=-=-=-=-=-=-=-=-=-=-=-=-=-=-=-=-=-=-=-=-=
%        LOADING PACKAGES
%-=-=-=-=-=-=-=-=-=-=-=-=-=-=-=-=-=-=-=-=-=-=-=-=
\usepackage[utf8]{inputenc}

\usepackage{chronology}

\renewcommand{\event}[3][e]{%
  \pgfmathsetlength\xstop{(#2-\theyearstart)*\unit}%
  \ifx #1e%
    \draw[fill=black,draw=none,opacity=0.5]%
      (\xstop, 0) circle (.2\unit)%
      node[opacity=1,rotate=45,right=.2\unit] {#3};%
  \else%
    \pgfmathsetlength\xstart{(#1-\theyearstart)*\unit}%
    \draw[fill=black,draw=none,opacity=0.5,rounded corners=.1\unit]%
      (\xstart,-.1\unit) rectangle%
      node[opacity=1,rotate=45,right=.2\unit] {#3} (\xstop,.1\unit);%
  \fi}%

%-=-=-=-=-=-=-=-=-=-=-=-=-=-=-=-=-=-=-=-=-=-=-=-=
%        BEAMER OPTIONS
%-=-=-=-=-=-=-=-=-=-=-=-=-=-=-=-=-=-=-=-=-=-=-=-=

%\setbeameroption{show notes}

%-=-=-=-=-=-=-=-=-=-=-=-=-=-=-=-=-=-=-=-=-=-=-=-=
%
%	PRESENTATION INFORMATION
%
%-=-=-=-=-=-=-=-=-=-=-=-=-=-=-=-=-=-=-=-=-=-=-=-=

\title{Introduction to Qt C++ Framework and Arduino}
\subtitle{Supplementary course}
%\date{\small{\jobname}}
\date{}
\author{\texttt{by Manash, EEE-2K12, KUET}}
\institute{\textit{Khulna University of Engineering \& Technology}}

\hypersetup{
pdfauthor = {Manash Kumar Mandal: manashmndl@gmail.com},
pdfsubject = {},
pdfkeywords = {},
pdfmoddate= {D:\pdfdate},
pdfcreator = {}
}

\begin{document}



%-=-=-=-=-=-=-=-=-=-=-=-=-=-=-=-=-=-=-=-=-=-=-=-=
%
%	TITLE PAGE
%
%-=-=-=-=-=-=-=-=-=-=-=-=-=-=-=-=-=-=-=-=-=-=-=-=

\maketitle


\begin{frame}{Overview}
    \tableofcontents
\end{frame}



%-=-=-=-=-=-=-=-=-=-=-=-=-=-=-=-=-=-=-=-=-=-=-=-=
%
%	TABLE OF CONTENTS: OVERVIEW
%
%-=-=-=-=-=-=-=-=-=-=-=-=-=-=-=-=-=-=-=-=-=-=-=-=

\section{C++ Primer}

%-=-=-=-=-=-=-=-=-=-=-=-=-=-=-=-=-=-=-=-=-=-=-=-=
%	FRAME:
%-=-=-=-=-=-=-=-=-=-=-=-=-=-=-=-=-=-=-=-=-=-=-=-=

\begin{frame}[allowframebreaks]{Quick Review of C++}


\begin{block}{Topic Overview in C++ Primer Section}
	\begin{itemize}
	    \item Header and Source File
		\item Dynamic Memory Allocation 
		\item Avoiding Memory Leakage
		\item Struct
		\item Class
		\item Object Oriented Programming Basics
		\item Generics
		\item Distributing Class between a Header and a Source File
	\end{itemize}
\end{block}

\end{frame}


\subsection{Header and Source File}

\begin{frame}[allowframebreaks]{Header and Source file and their Application}

\begin{exampleblock}{Reason behind Header and Source Files}

The main reason would be for separating the interface from the implementation. 

The header declares \alert{"what"} a class (or whatever is being implemented) will do, while the cpp file defines \alert{"how"} it will perform those features.

\end{exampleblock}

\vspace{8em}

\begin{alertblock}{How splitting helps?}

\begin{itemize}
    \item {This reduces dependencies so that code that uses the header doesn't necessarily need to know all the details of the implementation and any other classes/headers needed only for that.}
    \item {This will reduce compilation times and also the amount of recompilation needed when something in the implementation changes.}
\end{itemize}

\end{alertblock}

\end{frame}

\subsection{How to Create a Header File and Use it}

\begin{frame}{How to Create a Header file and use it}

Following these steps will do,

\begin{enumerate}
    \item {In your C++ project directory, create a file with extension \alert{.h}, example \alert{printArray.h}}
    \item {Now you need to use \alert{Guard Block} with the similar name as the header file}
    \item {Within the Guard Block put your code}
    \item {In your source file, call the header file using \alert{#include} directive}
    \end{enumerate}

\end{frame}

\subsection{Header File Example}

\begingroup
\setbeamercolor{background canvas}{bg=\cnLightBlue}
\begin{frame}[containsverbatim]{Simple.h}

Let's assume we have created a \alert{Simple.h} header file in our project directory

\begin{verbatim}
    //Let's say the name of the header is Simple.h
    #ifndef SIMPLE_H_
    #define SIMPLE_H_
    
    //Adding the standard library
    #include <stdio.h>
    int add(int a, int b);
    void print(char* a_string);
    #endif
\end{verbatim}

\end{frame}
\endgroup

\begingroup
\setbeamercolor{background canvas}{bg=\cnLightGreen}
\begin{frame}[containsverbatim]{Simple.c}

Now we have to implement the functions in this \alert{Simple.c} file


\begin{verbatim}
//Now adding our simple.h
#include "Simple.h"
    
//Implementation of the functions
int add(int a, int b) { return a + b; }
void print(char *a_string) { printf("%s", a_string); }

\end{verbatim}

\end{frame}
\endgroup

\begingroup
\setbeamercolor{background canvas}{bg=\cnLightRed}
\begin{frame}[containsverbatim]{main.c}

Let's try out our very own \alert{Simple.h} header file

\begin{verbatim}
#include <stdio.h>
//Yeah, we are going to add only Simple.h
#include "Simple.h"

int main(void)
{
   int c = add(2, 3);
   char* str = "Welcome to the workshop!";
   print(str);
}
\end{verbatim}

\end{frame}
\endgroup

\subsection{Dynamic Memory Allocation}

\begin{frame}[allowframebreaks]{Dynamic Memory Allocation in C++}

\begin{exampleblock}{Why do we need Dynamic Memory Allocation?}

\begin{itemize}
\item Many times, you are not aware in advance how much memory you will need to store particular information in a defined variable and the size of required memory can be determined at run time.

\item You can allocate memory at run time within the heap for the variable of a given type using a special operator in C++ which returns the address of the space allocated. This operator is called \alert{new} operator.
    
\end{itemize}


\end{exampleblock}

\vspace{1em}

Memory allocation using \alert{new} operator 

\begin{verbatim}

int *integer;

integer = new int(10);

//Dereferencing the variable for getting actual value from the address

std::cout << *integer;

\end{verbatim}
Memory Allocation for \alert{Arrays}

\begin{verbatim}
#define ARRAY\_SIZE 20

char *char\_array;

char\_array = new char[ARRAY\_SIZE];

\end{verbatim}

\end{frame}

\begin{frame}[allowframebreaks]{Avoiding Memory Leakage}

\begin{exampleblock}{What is Memory Leakage?}

Memory leaks are blocks of storage that are allocated but never freed by a program, even when no longer in use. Misuse of dynamic memory allocation may lead to memory leakage.

\end{exampleblock}

\begin{alertblock}{How to prevent memory leakage?}

If you're programming in OOP paradigm, it is a good practice to write memory freeing statements in destructors, which will be discussed later on. By using \alert{delete} keyword, memory of a pointer can be freed.


\end{alertblock}

A simple example of freeing a pointer array

\begin{verbatim}

int *int\_array = new int[10];

for (int i = 0; i < 10; i++)\{

         int\_array[i] = i * 5;
    
\}

//[] was used for freeing the integer array

delete[] int\_array;

\end{verbatim}

\end{frame}

\subsection{Structures in C++}

\begin{frame}[allowframebreaks]{Composite Data Type : Struct}

A composite data type consists of various types of data. Let's say I have a \alert{Box}, and in the box I store my books, pens and a RC car. 

Now I can treat the box as a variable, within the \alert{Box} variable, book, pen and a RC car type variables reside too. But I can access those variables at any time by opening the box. 

For the sake of explanation, we can call the box a composite type object which carries fundamental objects like book, RC car and pens.

\vspace{2em}

\begin{exampleblock}{If the box were a \alert{Structure} in my program, how would I access those objects (pen, book, RC car) in it?}

\begin{verbatim}
    //Creating a box type structure [let's assume as soon as I create a Box variable, an RC car, a book and a pen gets created automatically. 

    Box box;
    
    //Accessing the objects within the box
    
    cout << box.RC\_car << endl;
    
    cout << box.pen << endl;
    
    cout << box.book << endl;
    
\end{verbatim}

\end{exampleblock}


\end{frame}

\begingroup
\setbeamercolor{background canvas}{bg=\cnLightGreen}

\begin{frame}[containsverbatim, allowframebreaks]{Struct Example}
Creating a \alert{struct}
\begin{verbatim}
    struct Box {
        char pen;
        char[30] book;
        int RC_car;
    };
    //main function
    int main(void){
        Box box;
        box.pen = 'M';
        box.book = "Algorithms by CLRS";
        box.RC_car = 10;
        cout << box.pen << endl 
             << box.book << endl
             << box.RC_car << endl;
    }
\end{verbatim}

\end{frame}

\endgroup

\subsection{Class}

\begin{frame}[allowframebreaks, containsverbatim]{Class}

\alert{Class} can also be considered as a composite data type like \alert{struct}. But class is an updated version of it. That is why class comes with a lot of extra features.

\begin{exampleblock}{Features of Class}
    \begin{itemize}
        \item{Data Encapsulation : Access Modifiers}
        \item{Inheritable}
        \item{Constructors}
        \item{Destructors}
        \item{Functoion Overloading}
        \item{Operator Overloading}
        \item{Supports polymorphism}
        \item{Supports generics}
        \item{...}
    \end{itemize}
\end{exampleblock}


Let's talk about some functionality of class

\alert{Constructors:}
Constructors are functions in a class sharing the same name with the class itself and do not return anything. But when the object is created, these functions are called automatically. 
\begin{verbatim}
    class Student {
    //private and public are access modifiers
    private:
        string name;
        int id;
    public:
        //Constructor
        Student(string n, string i){
            name = n;
            id = i;
        }
        //Class can have it's own methods (functions)
        void printDetails(void){
            cout << "Name: " << name
                 << endl << "ID: " << 
                 << id << endl;
        }
    };
    int main(){
        Student manash("Manash", 1203043);
        manash.printDetails(); }
\end{verbatim}

From the above example, I created a student variable and passed the parameters without calling any function. Then, by calling the \alert{printDetails()} function it can be confirmed that the private fields were assigned with the value which were passed when the student variable was created.

\vspace{15em}

\begin{block}{Another way of assigning values to fields}
    
    \begin{verbatim}
        class Student {
            private:
                string name;
                int id;
            public:
                //This one works the same way before
                Student (string n, int i) : name(n), id(i) {}
                ...
        };
    \end{verbatim}
    
\end{block}

    \begin{block}{Accessing Fields of a class}
        \begin{enumerate}
            \item{Same as struct but, private fields can not be accessed simply by '.' notation}
            \item{Public methods of a class can access the private fields, so the best practice is to use get-set method for retrieving and  assigning values}
        \end{enumerate}
        
    \end{block}
    

\end{frame}

\begin{frame}[allowframebreaks, containsverbatim]{Scope Resolution Operator (::)}

\alert{Functions} and \alert{Static Variables} can be defined within the class or outside the class. Here is an example of defining methods outside of class.

\alert{Outside of Class}
\begin{verbatim}
class Student {
    private:
        string name;
        int id;
    public:
        //Just the interface of the function
        Student(string name, int id);
        printDetails(void);
};

class Teacher {
    private:
        string subject;
    public:
        Teacher(string subject);
        void teach();
};

void Student::printDetails(void){
    cout << "Name: " << name << endl
         << "Id: " << id << endl;
}
    
Student::Student(string name, int id){
    this->name = name;
    this->id = id;
}

Teacher::Teacher(string subject){
    this->subject = subject;
}

void Teacher::teach(){
    cout << "I teach " << this->subject << endl;
}
    
\end{verbatim}

We may come to a conclusion that with the help of \alert{Scope Resolution Operator} we can identify the classes with corresponding functions. This Scope Resolution Operator is also helpful for namespace selection.

\end{frame}


\begin{frame}[containsverbatim, allowframebreaks]{Object Oriented Programming Basics}
\alert{Inheritance:}
Let's create a class called Dad.

\begin{verbatim}
class Dad{
    private:
        string sur_name;
        int wealth;
    public:
        Dad(string _surname, int _wealth){
            sur_name = _surname;
            wealth = _wealth;
        }
        
        void printDetails(void){
            cout << "Sur name: " << sur_name << endl
                 << "Wealth: " << wealth << endl;
            }
    };
\end{verbatim}

Now let's create a class named Son which will inherit class Dad

\begin{verbatim}
class Son : public Dad {
    public:
        Son(string sur_name, int wealth) :
        Dad(sur_name, wealth) {}
    }
\end{verbatim}

I haven't created any method called \alert{printDetails} in class \alert{Son}, still I can call the function and print details since class Son inherits all the fields and methods of class Dad.

\begin{verbatim}
//In main.cpp
int main(){
    Son son("sur_name", 10000);
    son.printDetails();
}
\end{verbatim}

\vspace{16em}

\begin{verbatim}
Son(string sur_name, int wealth) :
        Dad(sur_name, wealth) {}
\end{verbatim}

This statement means \alert{Son} class will call the superclass constructor and set the variables. This is a way of calling constructors of superclass.

\end{frame}

\endgroup

\section{'this' keyword}

\begin{frame}[containsverbatim, allowframebreaks]{Use of 'this' keyword}

\alert{'this'} keyword is used to refer the object itself.

\begin{verbatim}
class Dummy {
  public:
    bool isitme (Dummy& param);
};

bool Dummy::isitme (Dummy& param)
{
  if (&param == this) return true;
  else return false;
}

int main () {
  Dummy a;
  Dummy* b = &a;
  if ( b->isitme(a) )
    cout << "yes, &a is b\n";
  return 0;
}
\end{verbatim}

\end{frame}



\section{Generics}



\begin{frame}[containsverbatim, allowframebreaks]{Introduction to Generics}
In short, generics means independence of data type. Let's see how it works.

Let's assume I've given a task to write a console application in C++ to add two data types. These data types may be float, double, integer or string (for string this addition operation can be called concatenation) maybe. To accomplish the task, you may think that I need to do something like this! (using function overloading)
\begin{verbatim}
//Adds ints
int add(int a, int b) { return a + b; }

//You guessed it!
float add(float a, float b) { return a + b; }

double add(double a, double b) { return a + b; }

std::string add(std::string a, std::string b) {
    return a + b; 
}
//.... The list goes on and on
\end{verbatim}

Suppose there are hundreds of data types, then I need to write hundreds of these functions for each data type which is not feasible. By using C++ generic, I can write a template based function which will handle all type of data types! Let's see how it's done!

\begin{verbatim}
template <typename any_type>
any_type add (any_type a, any_type b)
{
    return a + b;
}
\end{verbatim}

Traditionally it's written in this form

\begin{verbatim}
template <typename T>
T add(T a, T b)
{
    return a + b;
}
\end{verbatim}

For the time being, let's believe that \alert{typename} keyword is used for defining general data type. Not only functions but also \alert{Class}es can also be generic.

\begin{block}{Standard Template Library Example (Class) : Vector}
Vectors are resizable arrays. Here is a short example.
\end{block}

\alert{Example:}
\begin{verbatim}
//this one can store ints
vector <int> int_vector;

vector <char> char_vector;

for (int i = 0; i < 100; i++){
    int_vector.push_back(i);
    char_vector.push_back(char(i));
}

//Printing the value
for (int i : int_vector){
    cout << i << endl;
}

for (char a : char_vector){
    cout << a << endl;
}

\end{verbatim}
\end{frame}

\begin{frame}[containsverbatim, allowframebreaks]{Distributing Class between Header and Source File}

This concept will be useful for working with Qt Framework. Because, it is easier to maintain codes with separate files rather than one file containing all the codes. Let's see how can we split class function definitions and interface between header and source file.

\alert{StudentProfile.h}
\begin{verbatim}
#ifndef STUDENTPROFILE_H
#define STUDENTPROFILE_H
#include <iostream>

using std::string;

class StudentProfile
{
private:
    string name;
    int id;
    float cgpa;
    static int NUMBER_STUDENT_PROFILE_CREATED;
public:
    StudentProfile(string name, int id, float cg);
    string getName(void) const;
    int getId(void) const;
    float getCGPA(void) const;
    static int getTotalProfileCount(void);
};

#endif // STUDENTPROFILE_H
\end{verbatim}

Here \alert{static} variables are called Class Variables, because these variables do not belong to individual object of class, so this can not be accessed via object of a class. Static variables must be accessed using the Class itself. 

\alert{NUMBER\_STUDENT\_PROFILE\_CREATED} static variable was used to keep the count of the number of profile.

\alert{const} qualifier ensure that the returned value will not be changed under any circumstances. It may be considered as a fail-safe method of keep value unchanged.

\alert{getTotalProfileCount} static function was created to gain access to the private static field \alert{NUMBER\_STUDENT\_PROFILE\_CREATED}.

We know how to write function definition outside of class, I am going to apply the same method to write the source file.

\alert{StudentProfile.cpp}

\begin{verbatim}

#include "StudentProfile.h"

int StudentProfile::NUMBER_STUDENT_PROFILE_CREATED = 0;

StudentProfile::StudentProfile(string name, int id, float cg){
    this->name = name;
    this->id = id;
    this->cgpa = cg;
    //Since constructors are called automatically, if a object is created we can increase the profile count by 1 to keep track the total number of student profile
    StudentProfile::NUMBER_STUDENT_PROFILE_CREATED++;
}

string StudentProfile::getName () const {
    return this->name;
}

int StudentProfile::getId () const {
    return this->id;
}

float StudentProfile::getCGPA () const {
    return this->cgpa;
}

int StudentProfile::getTotalProfileCount() {
    return StudentProfile::NUMBER_STUDENT_PROFILE_CREATED;
}
\end{verbatim}

\alert{main.cpp}

\begin{verbatim}
int main(int argc, char *argv[])
{
    StudentProfile profile1("name1", 1, 1.00);
    StudentProfile profile2("name2", 2, 2.00);
    cout << StudentProfile::getTotalProfileCount () << endl;
}
\end{verbatim}



\end{frame}


\end{document}
