% 	Name		:: 	sthlm Beamer Theme  HEAVILY based on the hsrmbeamer theme (Benjamin Weiss)
%	Author		:: 	Mark Hendry Olson (mark@hendryolson.com)
%	Created		::	2013-07-31
%	Updated		::	June 18, 2015 at 08:45
%	Version		:: 	1.0.2
%	Email		:: 	hendryolson@gmail.com
%	Website		:: 	http://v42.com
%
% 	License		:: 	This file may be distributed and/or modified under the
%                  	GNU Public License.
%
%	Description	::	This presentation is a demonstration of the sthlm beamer
%					theme, which is HEAVILY based on the HSRM beamer theme created by Benjamin Weiss
%					(benjamin.weiss@student.hs-rm.de), which can be found on GitHub
%					<https://github.com/hsrmbeamertheme/hsrmbeamertheme>.


%-=-=-=-=-=-=-=-=-=-=-=-=-=-=-=-=-=-=-=-=-=-=-=-=
%
%        LOADING DOCUMENT
%
%-=-=-=-=-=-=-=-=-=-=-=-=-=-=-=-=-=-=-=-=-=-=-=-=

\documentclass[newPxFont]{beamer}
\usetheme{sthlm}
\usepackage{listings}
%\usecolortheme{sthlmv42}

%-=-=-=-=-=-=-=-=-=-=-=-=-=-=-=-=-=-=-=-=-=-=-=-=
%        LOADING PACKAGES
%-=-=-=-=-=-=-=-=-=-=-=-=-=-=-=-=-=-=-=-=-=-=-=-=
\usepackage[utf8]{inputenc}

\usepackage{chronology}

\renewcommand{\event}[3][e]{%
  \pgfmathsetlength\xstop{(#2-\theyearstart)*\unit}%
  \ifx #1e%
    \draw[fill=black,draw=none,opacity=0.5]%
      (\xstop, 0) circle (.2\unit)%
      node[opacity=1,rotate=45,right=.2\unit] {#3};%
  \else%
    \pgfmathsetlength\xstart{(#1-\theyearstart)*\unit}%
    \draw[fill=black,draw=none,opacity=0.5,rounded corners=.1\unit]%
      (\xstart,-.1\unit) rectangle%
      node[opacity=1,rotate=45,right=.2\unit] {#3} (\xstop,.1\unit);%
  \fi}%

%-=-=-=-=-=-=-=-=-=-=-=-=-=-=-=-=-=-=-=-=-=-=-=-=
%        BEAMER OPTIONS
%-=-=-=-=-=-=-=-=-=-=-=-=-=-=-=-=-=-=-=-=-=-=-=-=

%\setbeameroption{show notes}

%-=-=-=-=-=-=-=-=-=-=-=-=-=-=-=-=-=-=-=-=-=-=-=-=
%
%	PRESENTATION INFORMATION
%
%-=-=-=-=-=-=-=-=-=-=-=-=-=-=-=-=-=-=-=-=-=-=-=-=

\title{Introduction to Qt C++ Framework and Arduino}
\subtitle{Supplementary course}
%\date{\small{\jobname}}
\date{}
\author{\texttt{by Manash, EEE-2K12, KUET}}
\institute{\textit{Khulna University of Engineering \& Technology}}

\hypersetup{
pdfauthor = {Manash Kumar Mandal: manashmndl@gmail.com},
pdfsubject = {},
pdfkeywords = {},
pdfmoddate= {D:\pdfdate},
pdfcreator = {}
}

\begin{document}



%-=-=-=-=-=-=-=-=-=-=-=-=-=-=-=-=-=-=-=-=-=-=-=-=
%
%	TITLE PAGE
%
%-=-=-=-=-=-=-=-=-=-=-=-=-=-=-=-=-=-=-=-=-=-=-=-=

\maketitle


\begin{frame}{Overview}
    \tableofcontents
\end{frame}



%-=-=-=-=-=-=-=-=-=-=-=-=-=-=-=-=-=-=-=-=-=-=-=-=
%
%	TABLE OF CONTENTS: OVERVIEW
%
%-=-=-=-=-=-=-=-=-=-=-=-=-=-=-=-=-=-=-=-=-=-=-=-=

\section{C++ Primer}

%-=-=-=-=-=-=-=-=-=-=-=-=-=-=-=-=-=-=-=-=-=-=-=-=
%	FRAME:
%-=-=-=-=-=-=-=-=-=-=-=-=-=-=-=-=-=-=-=-=-=-=-=-=

\begin{frame}[allowframebreaks]{Quick Review of C++}


\begin{block}{Topic Overview in C++ Primer Section}
	\begin{itemize}
	    \item Header and Source File
		\item Dynamic Memory Allocation 
		\item Avoiding Memory Leakage
		\item Struct
		\item Class
		\item Object Oriented Programming Basics
		\item Generics
		\item Namespace
		\item Distributing Class between a Header and a Source File
	\end{itemize}
\end{block}

\end{frame}


\subsection{Header and Source File}

\begin{frame}[allowframebreaks]{Header and Source file and their Application}

\begin{exampleblock}{Reason behind Header and Source Files}

The main reason would be for separating the interface from the implementation. 

The header declares \alert{"what"} a class (or whatever is being implemented) will do, while the cpp file defines \alert{"how"} it will perform those features.

\end{exampleblock}

\vspace{8em}

\begin{alertblock}{How splitting helps?}

\begin{itemize}
    \item {This reduces dependencies so that code that uses the header doesn't necessarily need to know all the details of the implementation and any other classes/headers needed only for that.}
    \item {This will reduce compilation times and also the amount of recompilation needed when something in the implementation changes.}
\end{itemize}

\end{alertblock}

\end{frame}

\subsection{How to Create a Header File and Use it}

\begin{frame}{How to Create a Header file and use it}

Following these steps will do,

\begin{enumerate}
    \item {In your C++ project directory, create a file with extension \alert{.h}, example \alert{printArray.h}}
    \item {Now you need to use \alert{Guard Block} with the similar name as the header file}
    \item {Within the Guard Block put your code}
    \item {In your source file, call the header file using \alert{#include} directive}
    \end{enumerate}

\end{frame}

\subsection{Header File Example}

\begingroup
\setbeamercolor{background canvas}{bg=\cnLightBlue}
\begin{frame}[containsverbatim]{Simple.h}

Let's assume we have created a \alert{Simple.h} header file in our project directory

\begin{verbatim}
    //Let's say the name of the header is Simple.h
    #ifndef SIMPLE_H_
    #define SIMPLE_H_
    
    //Adding the standard library
    #include <stdio.h>
    int add(int a, int b);
    void print(char* a_string);
    #endif
\end{verbatim}

\end{frame}
\endgroup

\begingroup
\setbeamercolor{background canvas}{bg=\cnLightGreen}
\begin{frame}[containsverbatim]{Simple.c}

Now we have to implement the functions in this \alert{Simple.c} file


\begin{verbatim}
//Now adding our simple.h
#include "Simple.h"
    
//Implementation of the functions
int add(int a, int b) { return a + b; }
void print(char *a_string) { printf("%s", a_string); }

\end{verbatim}

\end{frame}
\endgroup

\begingroup
\setbeamercolor{background canvas}{bg=\cnLightRed}
\begin{frame}[containsverbatim]{main.c}

Let's try out our very own \alert{Simple.h} header file

\begin{verbatim}
#include <stdio.h>
//Yeah, we are going to add only Simple.h
#include "Simple.h"

int main(void)
{
   int c = add(2, 3);
   char* str = "Welcome to the workshop!";
   print(str);
}
\end{verbatim}

\end{frame}
\endgroup

\subsection{Dynamic Memory Allocation}

\begin{frame}[allowframebreaks]{Dynamic Memory Allocation in C++}

\begin{exampleblock}{Why do we need Dynamic Memory Allocation?}

\begin{itemize}
\item Many times, you are not aware in advance how much memory you will need to store particular information in a defined variable and the size of required memory can be determined at run time.

\item You can allocate memory at run time within the heap for the variable of a given type using a special operator in C++ which returns the address of the space allocated. This operator is called \alert{new} operator.
    
\end{itemize}


\end{exampleblock}

\vspace{1em}

Memory allocation using \alert{new} operator 

\begin{verbatim}

int *integer;

integer = new int(10);

//Dereferencing the variable for getting actual value from the address

std::cout << *integer;

\end{verbatim}
Memory Allocation for \alert{Arrays}

\begin{verbatim}
#define ARRAY\_SIZE 20

char *char\_array;

char\_array = new char[ARRAY\_SIZE];

\end{verbatim}

\end{frame}

\begin{frame}[allowframebreaks]{Avoiding Memory Leakage}

\begin{exampleblock}{What is Memory Leakage?}

Memory leaks are blocks of storage that are allocated but never freed by a program, even when no longer in use. Misuse of dynamic memory allocation may lead to memory leakage.

\end{exampleblock}

\begin{alertblock}{How to prevent memory leakage?}

If you're programming in OOP paradigm, it is a good practice to write memory freeing statements in destructors, which will be discussed later on. By using \alert{delete} keyword, memory of a pointer can be freed.


\end{alertblock}

A simple example of freeing a pointer array

\begin{verbatim}

int *int\_array = new int[10];

for (int i = 0; i < 10; i++)\{

         int\_array[i] = i * 5;
    
\}

//[] was used for freeing the integer array

delete[] int\_array;

\end{verbatim}

\end{frame}

\subsection{Structures in C++}

\begin{frame}[allowframebreaks]{Composite Data Type : Struct}

A composite data type consists of various types of data. Let's say I have a \alert{Box}, and in the box I store my books, pens and a RC car. 

Now I can treat the box as a variable, within the \alert{Box} variable, book, pen and a RC car type variables reside too. But I can access those variables at any time by opening the box. 

For the sake of explanation, we can call the box a composite type object which carries fundamental objects like book, RC car and pens.

\vspace{2em}

\begin{exampleblock}{If the box were a \alert{Structure} in my program, how would I access those objects (pen, book, RC car) in it?}

\begin{verbatim}
    //Creating a box type structure [let's assume as soon as I create a Box variable, an RC car, a book and a pen gets created automatically. 

    Box box;
    
    //Accessing the objects within the box
    
    cout << box.RC\_car << endl;
    
    cout << box.pen << endl;
    
    cout << box.book << endl;
    
\end{verbatim}

\end{exampleblock}


\end{frame}

\begingroup
\setbeamercolor{background canvas}{bg=\cnLightGreen}

\begin{frame}[containsverbatim, allowframebreaks]{Struct Example}
Creating a \alert{struct}
\begin{verbatim}
    struct Box {
        char pen;
        char[30] book;
        int RC_car;
    }
    //main function
    int main(void){
        Box box;
        box.pen = 'M';
        box.book = "Algorithms by CLRS";
        box.RC_car = 10;
        cout << box.pen << endl 
             << box.book << endl
             << box.RC_car << endl;
    }
\end{verbatim}

\end{frame}

\endgroup

\subsection{Class}

\begin{frame}[allowframebreaks]{Class}

\alert{Class} can also be considered as a composite data type like \alert{struct}. But it is an updated version of struct. That is why class comes with a lot of extra features.

\begin{exampleblock}{Features of Class}
    \begin{itemize}
        \item{Data Encapsulation : Access Modifiers}
        \item{Inheritable}
        \item{Supports polymorphism}
        \item{Supports generics}
    \end{itemize}
\end{exampleblock}

\end{frame}

\end{document}
